\documentclass[journal,12pt,twocolumn]{IEEEtran}
\usepackage{graphicx}
\usepackage{paralist}
\usepackage{setspace}
\usepackage{gensymb}
\singlespacing
\usepackage[cmex10]{amsmath}
\usepackage{amsthm}
\usepackage{amsmath}
\usepackage{amssymb}
\usepackage{mathrsfs}
\usepackage{txfonts}
\usepackage{stfloats}
\usepackage{bm}
\usepackage{cite}
\usepackage{cases}
\usepackage{subfig}
\usepackage{longtable}
\usepackage{multirow}

\usepackage{enumitem}
\usepackage{mathtools}
\usepackage{steinmetz}
\usepackage{tikz}
\usepackage{circuitikz}
\usepackage{verbatim}
\usepackage{tfrupee}
\usepackage[breaklinks=true]{hyperref}
\usepackage{graphicx}
\usepackage{tkz-euclide}

\usetikzlibrary{calc,math}
\usepackage{listings}
    \usepackage{color}                                            %%
    \usepackage{array}                                            %%
    \usepackage{longtable}                                        %%
    \usepackage{calc}                                             %%
    \usepackage{multirow}                                         %%
    \usepackage{hhline}                                           %%
    \usepackage{ifthen}                                           %%
    \usepackage{lscape}     
\usepackage{multicol}
\usepackage{chngcntr}

\DeclareMathOperator*{\Res}{Res}

\renewcommand\thesection{\arabic{section}}
\renewcommand\thesubsection{\thesection.\arabic{subsection}}
\renewcommand\thesubsubsection{\thesubsection.\arabic{subsubsection}}

\renewcommand\thesectiondis{\arabic{section}}
\renewcommand\thesubsectiondis{\thesectiondis.\arabic{subsection}}
\renewcommand\thesubsubsectiondis{\thesubsectiondis.\arabic{subsubsection}}


\hyphenation{op-tical net-works semi-conduc-tor}
\def\inputGnumericTable{}                                 %%

\lstset{
%language=C,
frame=single, 
breaklines=true,
columns=fullflexible
}
\begin{document}


\newtheorem{theorem}{Theorem}[section]
\newtheorem{problem}{Problem}
\newtheorem{proposition}{Proposition}[section]
\newtheorem{lemma}{Lemma}[section]
\newtheorem{corollary}[theorem]{Corollary}
\newtheorem{example}{Example}[section]
\newtheorem{definition}[problem]{Definition}

\newcommand{\BEQA}{\begin{eqnarray}}
\newcommand{\EEQA}{\end{eqnarray}}
\newcommand{\define}{\stackrel{\triangle}{=}}
\bibliographystyle{IEEEtran}
\raggedbottom
\setlength{\parindent}{0pt}
\providecommand{\mbf}{\mathbf}
\providecommand{\pr}[1]{\ensuremath{\Pr\left(#1\right)}}
\providecommand{\qfunc}[1]{\ensuremath{Q\left(#1\right)}}
\providecommand{\sbrak}[1]{\ensuremath{{}\left[#1\right]}}
\providecommand{\lsbrak}[1]{\ensuremath{{}\left[#1\right.}}
\providecommand{\rsbrak}[1]{\ensuremath{{}\left.#1\right]}}
\providecommand{\brak}[1]{\ensuremath{\left(#1\right)}}
\providecommand{\lbrak}[1]{\ensuremath{\left(#1\right.}}
\providecommand{\rbrak}[1]{\ensuremath{\left.#1\right)}}
\providecommand{\cbrak}[1]{\ensuremath{\left\{#1\right\}}}
\providecommand{\lcbrak}[1]{\ensuremath{\left\{#1\right.}}
\providecommand{\rcbrak}[1]{\ensuremath{\left.#1\right\}}}
\theoremstyle{remark}
\newtheorem{rem}{Remark}
\newcommand{\sgn}{\mathop{\mathrm{sgn}}}
\providecommand{\abs}[1]{\left\vert#1\right\vert}
\providecommand{\res}[1]{\Res\displaylimits_{#1}} 
\providecommand{\norm}[1]{\left\lVert#1\right\rVert}
%\providecommand{\norm}[1]{\lVert#1\rVert}
\providecommand{\mtx}[1]{\mathbf{#1}}
\providecommand{\mean}[1]{E\left[ #1 \right]}
\providecommand{\fourier}{\overset{\mathcal{F}}{ \rightleftharpoons}}
%\providecommand{\hilbert}{\overset{\mathcal{H}}{ \rightleftharpoons}}
\providecommand{\system}{\overset{\mathcal{H}}{ \longleftrightarrow}}
	%\newcommand{\solution}[2]{\textbf{Solution:}{#1}}
\newcommand{\comb}[2]{{}^{#1}\mathrm{C}_{#2}}
\newcommand{\solution}{\noindent \textbf{Solution: }}
\newcommand{\cosec}{\,\text{cosec}\,}
\newcommand{\cosec}{}
\providecommand{\dec}[2]{\ensuremath{\overset{#1}{\underset{#2}{\gtrless}}}}
\newcommand{\myvec}[1]{\ensuremath{\begin{pmatrix}#1\end{pmatrix}}}
\newcommand{\mydet}[1]{\ensuremath{\begin{vmatrix}#1\end{vmatrix}}}
\numberwithin{equation}{subsection}
\makeatletter
\@addtoreset{figure}{problem}
\makeatother
\let\StandardTheFigure\thefigure
\let\vec\mathbf
\renewcommand{\thefigure}{\theproblem}
\def\putbox#1#2#3{\makebox[0in][l]{\makebox[#1][l]{}\raisebox{\baselineskip}[0in][0in]{\raisebox{#2}[0in][0in]{#3}}}}
     \def\rightbox#1{\makebox[0in][r]{#1}}  
     \def\centbox#1{\makebox[0in]{#1}}
     \def\topbox#1{\raisebox{-\baselineskip}[0in][0in]{#1}}
     \def\midbox#1{\raisebox{-0.5\baselineskip}[0in][0in]{#1}}
\vspace{3cm}
\title{Assignment 6}
\author{Gaureesha Kajampady - EP20BTECH11005}
\maketitle  
\newpage
\bigskips
\renewcommand{\thefigure}{\theenumi}
\renewcommand{\thetable}{\theenumi}
Download all python codes from 
\begin{lstlisting}
https://github.com/gaureeshk/assignment6_1/tree/main/Codes/assignment6.py
\end{lstlisting}
%
Download latex-tikz codes from 
%
\begin{lstlisting}
https://github.com/gaureeshk/assignment6_1/blob/main/assignment6.tex
\end{lstlisting}
\section{Problem}
(CSIR UGC NET EXAM (Dec 2012), Q.51)\\
Suppose $X_{1}, X_{2}, X_{3}, X_{4}$ are i.i.d random variables taking values 1 and -1 with probability $\frac{1}{2}$ each. Then $E(X_{1}+X_{2}+X_{3}+X_{4})^{4}$ equals
\begin{enumerate}
    \item{40}\\
    \item{76}\\
     \item{16}\\
      \item{12}\\
\end{enumerate}
\section{Solution} 
Using multinomial expansion for$(X_{1}+X_{2}+X_{3}+X_{4})^{4}$,\\
\begin{multline}
    (X_{1}+X_{2}+X_{3}+X_{4})^{4}=\frac{4!}{4!}\sum_{i=1}^{4}X_{i}^{4}+\frac{4!}{3!1!}\sum_{i=1}^{4}\!\sum_{j=1(j \neq i)}^{4}X_{i}^{3}X_{j}\\ \hspace{1 in}+\frac{4!}{2!2!}\sum_{i=1}^{4}\!\sum_{j=1(j \neq i)}^{4}X_{i}^{2}X_{j}^{2}\\
\end{multline}
Since expectation is a linear function,
\begin{multline}
    E(X_{1}+X_{2}+X_{3}+X_{4})^{4}=\sum_{i=1}^{4}E(X_{i}^{4})+4\!\!\sum_{i=1}^{4}\!\sum_{j=1(j \neq i)}^{4}\!\!\!E(X_{i}^{3}X_{j})\\ \hspace{1 in}+6\sum_{i=1}^{4}\!\sum_{j=1(j \neq i)}^{4}E(X_{i}^{2}X_{j}^{2})\\
\end{multline}
Since $X_{i}$ can only be +1 or -1, $X_{i}^{4}$ is always 1. Hence 
\begin{align}
E(X_{i}^{4})=1
\end{align}\\
Similarly for $X_{i}^{3}X_{j}$ term we get the following possibilities\\
\begin{table}[h]
\begin{tabular}{lll}
X_{i} & X_{j} & X_{i}^{3}X{j}  \\
1   & 1 & 1 \\
-1   & -1 & 1 \\
-1 & 1 & -1 \\
1 & -1 & -1\\
\end{tabular}
\end{table}\\
Hence $X_{i}^{3}X_{j}$ has values 1 and -1 with probability $\frac{1}{2}$ each.  
\begin{align}
   \implies E(X_{i}^{3}X_{j})=\frac{1}{2}\times1 + \frac{1}{2}\times-1=0
\end{align}
$X_{i}^{2}X_{j}^{2}$ is always 1 hence $E(X_{i}^{2}X_{j}^{2})$=1\\
To find the number of $X_{i}^{2}X_{j}^{2}$ terms, \\
There are 4 random variables and we need to select 2 (i and j) without any order (because $X_{i}^{2}X_{j}^{2}$= $X_{j}^{2}X_{i}^{2}$)\\
\begin{align}
    \implies \text{total number of terms }=& \comb{4}{2}\\
    &=6\\
    \!\!\!\!\!\!\implies E(X_{1}+X_{2}+X_{3}+X_{4})^{4}&=4\times 1+0+6\times6\times1\\
                                         &=40
\end{align}
Hence option 1 is correct.
\end{document}





















